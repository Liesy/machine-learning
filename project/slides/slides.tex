\documentclass[aspectratio=43, 12pt, utf8, mathserif]{ctexbeamer} %aspectratio=169
%导言区
%\usepackage{ctex}
\usepackage{amsmath, amsfonts, amssymb, amsthm}
\usepackage{graphicx}
\usepackage{fontspec}
\usepackage{xeCJK}
\usepackage{ulem} %解决下划线换行紊乱
\usepackage{caption} %添加标题
\usepackage{subfigure}
\usepackage{theorem}
\usepackage[backend=bibtex,style=numeric-comp,sorting=none]{biblatex} %不列出所有作者
%\usepackage[backend=bibtex,sorting=none,maxnames=9,minnames=3]{biblatex} %列出所有作者
\addbibresource{ref.bib} %BibTeX数据文件及位置
\setbeamerfont{footnote}{size=\tiny} %设置脚注引用文献的字体大小
\setbeamertemplate{bibliography item}[text] %设置参考文献图标样式数字标号
\usepackage{multicol} %分栏
\usepackage{syntonly} %只编译文件是否成功,省时省力
%\syntaxonly %不注释代表只编译是否成功
%\usepackage[marginal]{footmisc} %首页添加脚注无缩进
%\renewcommand{\thefootnote}{} %首页添加脚注无编号
\usepackage{enumerate}
\usepackage{subfigure}
\usepackage{theorem}

%\setbeamertemplate{navigation symbols}{} %取消导航
\setCJKmainfont{SimHei} %中文用黑体

% 设置页脚格式
\setbeamertemplate{footline}{%
	\leavevmode%
	\hbox{%
		\begin{beamercolorbox}[wd=0.3\paperwidth,ht=2.25ex,dp=1ex,center]{author in head/foot}%
		\usebeamerfont{author in head/foot}\insertshortinstitute \quad \insertshortauthor
		\end{beamercolorbox}%
		\begin{beamercolorbox}[wd=0.4\paperwidth,ht=2.25ex,dp=1ex,center]{title in head/foot}%
			\usebeamerfont{title in head/foot}\insertshorttitle
		\end{beamercolorbox}%
		\begin{beamercolorbox}[wd=0.3\paperwidth,ht=2.25ex,dp=1ex,center]{date in head/foot}%
			\usebeamerfont{date in head/foot}\insertshortdate{} \quad
			\insertframenumber{} / \inserttotalframenumber
	\end{beamercolorbox}}%
	\vskip0pt%
}


%使用的主题样式和主题色
%----主题----
% \usetheme{AnnArbor}
\usetheme{Antibes}
%\usetheme{Berlin} %不显示底栏
% \usetheme{cambridgeUS}
%\usetheme{Darmstadt}
%\usetheme{Dresden} %不显示底栏
%\usetheme{Frankfurt}
%\usetheme{Ilmenau}
%\usetheme{Hannover}
% \usetheme{Berkeley}
%\usetheme{EastLansing}


%----设置颜色、外框颜色等----
\useinnertheme{circles}
\useoutertheme{miniframes} %tree、miniframes
%\usecolortheme{spruce} % 该色调很多不显示底栏
\usecolortheme{dolphin} % 该色调很多不显示底栏
%\usecolortheme{crane}
\usefonttheme{serif} %已有的字体default professionalfonts serif structurebold structureitalicserif structuresmallcapsserif

\definecolor{zdyblue}{RGB}{19,63,127} %千万不能写rgb
\definecolor{zdyred}{RGB}{184,0,0} %千万不能写rgb

% 设置用acrobat打开就会全屏显示
% \hypersetup{pdfpagemode=FullScreen}

%-------------开始-------------------
\begin{document}

\title{\bf 心跳信号分类预测}
% \subtitle{\bf 2021年春 机器学习与模式识别 课程项目}
\author{李阳}
\institute[智能19]{
	山东大学计算机科学与技术学院 \\ \vspace*{0.1cm}
	2019级智能班
}
\date{\today}

\begin{frame}
    % \maketitle
    \titlepage    
\end{frame}

% \begin{frame}
% 	\frametitle{总目录}
% 	\begin{multicols}{2}
% 		\tableofcontents[hideallsubsections]
% 	\end{multicols}
% %	\tableofcontents[hideallsubsections]
% \end{frame}

\section{问题分析}
\begin{frame}
    \frametitle{问题分析}
    \begin{enumerate}
        \item 多分类问题
        \item 数据量大
        \item 特征是一串心跳序列
        \item 结果提交的是4种不同心跳信号预测的概率,而非单一的预测所属分类
    \end{enumerate}
\end{frame}

\begin{frame}
    \frametitle{多分类问题}
    内容
\end{frame}

\begin{frame}
    \frametitle{数据量大}
    内容
\end{frame}

\begin{frame}
    \frametitle{特征}
    内容
\end{frame}

\begin{frame}
    \frametitle{结果提交}
    内容
\end{frame}

\section{数据处理}
\begin{frame}
    \frametitle{数据处理}
    内容
\end{frame}

\section{方法选用}
\begin{frame}
    \frametitle{方法选用}
    \begin{itemize}
        \item AdaBoost
        \item 元分类器是支持向量机(SVM)的AdaBoost
        \item 随机森林 Random Forest
    \end{itemize}
\end{frame}

\begin{frame}
    \frametitle{AdaBoost}
    内容
\end{frame}

\begin{frame}
    \frametitle{AdaBoost + SVM}
    内容
\end{frame}

\begin{frame}
    \frametitle{Random Forest}
    内容
\end{frame}

\begin{frame}
    \frametitle{选取最优方案}
    内容
\end{frame}

\section{写在最后}
\begin{frame}
    \frametitle{参考内容}
    \begin{enumerate}
        \item \href{https://zhuanlan.zhihu.com/p/44695084}{随机森林 - Random Forest}
        \item \href{https://www.jiqizhixin.com/articles/2020-06-11-6}{决策树VS随机森林}
        \item \href{https://blog.csdn.net/baidu_31657889/article/details/93891552?utm_source=app&app_version=4.7.1}{利用AdaBoost元算法提高分类性能}
    \end{enumerate}
\end{frame}

\begin{frame}
	\zihao{-4}\centering{Thank you!}
\end{frame}

\end{document}